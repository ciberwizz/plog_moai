\documentclass[15pt,a4paper]{article}

\usepackage[portuguese]{babel}
\usepackage[utf8]{inputenc}
\usepackage{indentfirst}
\usepackage{graphicx}
\usepackage{verbatim}


\begin{document}

\setlength{\textwidth}{16cm}
\setlength{\textheight}{22cm}

\title{\Huge\textbf{Moai}\linebreak\linebreak\linebreak
\Large\textbf{Relatório Intercalar}\linebreak\linebreak
\includegraphics[height=6cm, width=7cm]{feup.pdf}\linebreak \linebreak
\Large{Mestrado Integrado em Engenharia Informática e Computação} \linebreak \linebreak
\Large{Programação em Lógica}\linebreak
}

\author{\textbf{Grupo 5:}\\ Paulo Jorge de Faria dos Reis - 080509037 \\ Miguel Rossi Seabra - 060509054 \\\linebreak\linebreak \\
 \\ Faculdade de Engenharia da Universidade do Porto \\ Rua Roberto Frias, s\/n, 4200-465 Porto, Portugal \linebreak\linebreak\linebreak
\linebreak\linebreak\vspace{1cm}}
%\date{4 de Novembro de 2012}
\maketitle
\thispagestyle{empty}

%************************************************************************************************
%************************************************************************************************

\newpage

\section*{Resumo}
Pretende-se com este relatório apresentar o procedimento seguido pelo grupo de trabalho no desenvolvimento em Prolog do jogo "Moai".
\\
Será disponibilizada informação relativamente ao jogo em termos de regras, descrição do tabuleiro, opções tomadas pelo grupo em pontos onde a informação do criador do daquele não é clara e também uma apresentação dos predicados de Prolog que vão ser utilizados para o desenvolvimento do projeto.

\newpage

\tableofcontents

%************************************************************************************************
%************************************************************************************************

%*************************************************************************************************
%************************************************************************************************

\newpage

\section{Introdução}
A implementação das regras de um jogo de tabuleiro recente, utilizando programação em lógica apresenta um desafio aliciante, ao mesmo tempo que possibilita o contacto e desenvolvimento de capacidades nesta área aos elementos envolvidos no desenvolvimento deste sistema. É também de ressalvar a criação de um programa funcional que permite a realização de partidas de ''Moai'' onde os intervenientes podem ser humanos ou computadores, e neste caso tendo o sistema capacidades mínimas de realizar jogadas válidas de acordo com as regras. A versão base apenas permitirá a interação em modo de texto através da consolo de Prolog, podendo no entanto ser providenciada a integração com interface gráfico.

%*************************************************************************************************
%************************************************************************************************

\section{Descrição do Problema}
O jogo foi criado por Rey Alicea em 2012 \cite{gamegeek}, e não existe qualquer referência histórica deste. Pelo que nos apercebemos o jogo encontra-se ainda num estádio de revisão do mesmo. A descrição das regras \cite{blog, gamegeek} não são claras o suficiente, deixando muito espaço para a interpretação das mesmas, assim a equipa de desenvolvimento das regras em Prolog decidiu utilizar a regras conforme se explica mais adiante.
\\
São necessários dois jogadores para realizar uma partida de "Moai".

\section{Arquitetura do Sistema}
Descrever em linhas gerais o sistema e os módulos que o constituem. Deve ser abordada a comunicação com o visualizador, que mesmo que ainda não esteja implementada, já deverá estar pensada. Assim, deve ser incluída a sintaxe das mensagens a trocar com o visualizador.

\section{Módulo de Lógica do Jogo}
Descrever o projecto e implementação do módulo Prolog, incluindo a forma de representação do estado do tabuleiro,  verificação do cumprimento das regras do jogo, determinação do final do jogo e cálculo das jogadas a realizar pelo computador  utilizando diversos níveis de jogo.

\subsection{Representação do Estado do Jogo} \textit{estado(?Tabuleiro).}
\subsection{Visualização do Estado do Jogo} \textit{visualiza\_estado(+Tabuleiro).}
\subsection{Validação de Jogadas} \textit{movimento\_valido(?Jogada, +Tabuleiro).}
\subsection{Execução de Jogadas}\textit{executa\_movimento(+Jogada, + Tabuleiro, -NovoTabuleiro).}
\subsection{Lista de Jogadas Válidas}\textit{lista\_jogadas(+Tabuleiro, -ListaJogadas).}
\subsection{Avaliação do Tabuleiro}\textit{avalia\_tabuleiro(+Tabuleiro, -Valor).}
\subsection{Final do Jogo} \textit{fim\_jogo(+Tabuleiro, -Vencedor).}
\subsection{Cálculo da Jogada do Computador}\textit{calcula\_jogada(+Nível, +Tabuleiro, -Jogada).}
\subsection{Recepção de mensagem do visualizador}\textit{recebe\_mensagem(+ Mensagem, - Resposta).}

\section{Interface com o Utilizador}
Não foi implementado o módulo de comunicação com o visualizador. A Interface com o utilizador é feita através de predicados do próprio Prolog.
Para iniciar um novo jogo deve-se utilizador o predicado \textit{new_game}, de seguida devemos informar as dimensões do tabuleiro que pretendemos utilizar, para tal devem ser introduzido um valor inteiro seguido de . (ponto) para cada uma das dimensões.
Uma vez que cada jogador pode ser controlado por um humano ou pela AI do computador, temos de responder cada um dos jogadores é humano ou não usando \textit{s} ou \textit{n}.
A partir deste ponto o jogo inicia-se começando pelo Jogador Branco e alternando com o Jogador Preto até que se alcance a condição de vitória (derrota) de um deles, terminando assim o jogo.

\section{Conclusões e Perspetivas de Desenvolvimento}
Utilizar o Prolog para definir as regras de um jogo é algo relativamente simples, sendo apenas necessário a definição de um conjunto de predicados (regras) que descrevam o que é ou não possível fazer no jogo. A comunicação com o utilizador é bem complicada de desenvolver neste contexto.
\\
A definição de regras passa pela elaboração de factos que definem condições de paragem e clausulas com predicados que permitem o processamento de casos gerais.


Que conclusões  retira deste projecto? Como poderia melhorar o trabalho desenvolvido?

\clearpage
\addcontentsline{toc}{section}{Bibliografia}
\renewcommand\refname{Bibliografia}
\bibliographystyle{plain}
\bibliography{myrefs}

\listoffigures

\newpage
\appendix
\section{Nome do Anexo A}
Código Prolog implementado devidamente comentado e outros elementos úteis que não sejam essenciais ao relatório.

\end{document}
